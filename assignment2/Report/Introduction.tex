\section{Introduction}
Our purpose in this assignment is to explore the uses of homographies, and apply this theory in order to perform various linear transformations. Projections of points from one 2d image onto another is the main focus.\newline

\section{Problem statement}

The assignment is split in four parts. Throughout the report, we have made an effort to address all of these assignments.\\

To help us along the way, we have been given the \textsl{SIGB-Tools.py} library, containing useful functions to help us.\\

In part one, the assignment is to track a person who is moving around in the atrium of the IT-university. To solve this assignment, we have been given the datafile \textsl{trackingdata.dat}, containing the data relevant to the person walking in atrium.\\

The second assignment is to project a texture onto a series of image sequences using texture mapping.\\

In the third assignment, we are tasked with exploring \textbf{camera calibration}, and will be using a image sequence of a chessboard pattern to calculate the camera matrix of the camera that made the recording.\\

In the fourth assignment we will explore \textbf{augmentation}. Here, we will make use of a webcam, calibrate it and project a nonexistent cube onto a surface defined by a chessboard pattern.
\newpage