\section{Finding the iris}
\subsection{Blob detection}
Our first approach to find the iris was to replicate the pupil detection method. By using different threshold values we tried to implement blob detection to find the center of the iris but this task was close to impossible. We found that the iris either merged with other features, like eyelashes and pupil, or disappeared with the rest.
\subsection{Edges and gradients} 
After the first approach failed we turned to gradients for iris detection. Boundaries in an image can be determined by a significant change of intensity between pixels. This change of intensity can be seen as a the slope, or gradient, at that point in the image. With this in mind we first calculated the gradients, gradient magnitudes, and gradient directions of the entire grey scale image to understand what the gradients in our sequences mean. In general, the properties of this gradients where as expected. In areas where boundaries are present also grater gradients are present and their direction is towards the white area. 
Knowing the characteristics of the gradients in the image sequences we used the \emph{getCircleSamples()} function from \emph{SIGBTools.py} to fit a circle around the center of the pupil acquired with blob detection. This circle was helpful to find the boundary of the pupil by analyzing the gradients along the normals between the center of the pupil and the edge of the circle. The same method was used to try and estimate the iris but the results where not good compared to the pupil detection. Even by trying to disregard pixels that where not aligned with the gradient curve the results where really affected by other features like the eyelashes.