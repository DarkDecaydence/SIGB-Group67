\section{Discussion}
	In this section we discuss and interpret the results for each part of the assignment.
\subsection{Texture Mapping}
A few challenges were faced to achieve the results for this part. The cube that was already defined in the file Assignment\_Cube.py was difficult to understand so we defined a cube for ourselves. Some of the code from the previous assignment stayed the same but a lot had to be or restructured to achieve the correct behavior for the assignment. A strange behavior when the cube is moved could not be solved. That is, when the cube rotates or moves in general it sometimes seems as if one of the faces covers the entire screen for less than a second but the source of this problem could not be found. Leaving aside these challenges, we have achieved a successful texture mapping for each face of the cube.

\subsection{Back-Face Culling}
A small set of challenges were met in this part. The most important of them was to find the correct implementation to get the points of each face of the cube in a counter clockwise orientation. Due to a confusion with the implementation we ended having two methods to calculate the normals. The main method we use is the one requested in the assignment. The second method determines the normal on the already projected cube and calculates the dot product between the normal vector and the vector (0,0,1), if the result is positive the face is drawn. Both methods work very good.

\subsection{Shading}
The video file "CubeNoShading.avi" shows the cube with no shading being applied. We observe that the cube follows the underlying grid very well, that the textures are being applied correctly, and that backface culling removes unwanted faces. We observe two types of mistake made by the program. Firstly, the cube can not be projected in the cases where the chessboard pattern cannot be located; this is a leftover flaw from the callibration process. Secondly, we observe irratic projections in cases where one of the faces fall parellel to the camera view vector. This is likely due the the homography being estimated from the four projected corner points failing when the four points are projected to line on a single line. In these cases there is not enough information to correctly determine the homography, and undefined results... result. This could have been helped by incorporating a detection of these cases into the backface culling algorithm to remove these faces; when they are parallel to the view vector, there is no need to project them.\\
The file "CubeFlatShading.avi" shows the cube with simple "flat" shading being applied, in which the normal of each point is equal to the normal of the face. We observe the expected sharp edges between faces and monotone shading across the face.\\
The file "CubePhongShading.avi" shows the cube with the full illumination algorithm being applied. We observe as expected the brightest point of the shading travelling across the cube as the camera moves.\\
As the diffuse shading is determined by the light position, and the specular shading is determined by the light- and the camera positions, and these, due to our constraints, are always equal, we observe that the brightest points in the two shading types always coincide. This would not necesarrily have been the case if the camera and light source where moving independently.