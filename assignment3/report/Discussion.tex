\section{Discussion}
	In this section we discuss and interpret the results for each part of the assignment.
\subsection{Texture Mapping}
A few challenges were faced to achieve the results for this part. The cube that was already defined in the file Assignment_Cube.py was difficult to understand so we defined a cube for ourselves. Some of the code from the previous assignment stayed the same but a lot had to be or restructured to achieve the correct behavior for the assignment. A strange behavior when the cube is moved could not be solved. That is, when the cube rotates or moves in general it sometimes seems as if one of the faces covers the entire screen for less than a second but the source of this problem could not be found. Leaving aside these challenges, we have achieved a successful texture mapping for each face of the cube.

\subsection{Back-Face Culling}
A small set of challenges were met in this part. The most important of them was to find the correct implementation to get the points of each face of the cube in a counter clockwise orientation. Due to a confusion with the implementation we ended having two methods to calculate the normals. The main method we use is the one requested in the assignment. The second method determines the normal on the already projected cube and calculates the dot product between the normal vector and the vector (0,0,1), if the result is positive the face is drawn. Both methods work very good.